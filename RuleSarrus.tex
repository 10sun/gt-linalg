% Mnemonic rule for matrix determinant 
% Author: Alain Matthes
\documentclass[]{article}
\usepackage[utf8]{inputenc}
\usepackage[upright]{fourier}
\usepackage{tikz}
%%%<
\usepackage{verbatim}
\usepackage[active,tightpage]{preview}
\PreviewEnvironment{tikzpicture}
\setlength\PreviewBorder{5pt}%
%%%>

\usetikzlibrary{matrix}
\usepackage{fullpage,amsmath}
\begin{document}






\begin{tikzpicture}[baseline=(A.center)]
\tikzset{node style ge/.style={circle}}
  \tikzset{BarreStyle/.style =   {opacity=.4,line width=4 mm,line cap=round,color=#1}}
    \tikzset{SignePlus/.style =   {above left,,opacity=1,circle,fill=#1!50}}
    \tikzset{SigneMoins/.style =   {below left,,opacity=1,circle,fill=#1!50}}
% les matrices
\matrix (A) [matrix of math nodes, nodes = {node style ge},,column sep=0 mm] 
{ a_{11} & a_{12} & a_{13} & a_{11} & a_{12} \\
  a_{21} & a_{22} & a_{23} & a_{21} & a_{22} \\
  a_{31} & a_{32} & a_{33}  & a_{31} & a_{32} \\
};

 \draw [BarreStyle=blue] (A-1-1.north west) node[SignePlus=blue] {$+$} to (A-3-3.south east) ;
 \draw [BarreStyle=blue] (A-1-2.north west) node[SignePlus=blue] {$+$} to (A-3-4.south east) ;
  \draw [BarreStyle=blue] (A-1-3.north west) node[SignePlus=blue] {$+$} to (A-3-5.south east) ;
 \draw [BarreStyle=red]  (A-3-1.south west) node[SigneMoins=red] {$-$} to (A-1-3.north east);
  \draw [BarreStyle=red]  (A-3-2.south west) node[SigneMoins=red] {$-$} to (A-1-4.north east);
   \draw [BarreStyle=red]  (A-3-3.south west) node[SigneMoins=red] {$-$} to (A-1-5.north east);
   \node[draw,text width=5cm] at (0,-2) {\tiny{(Modified from TikZ example of  Alain Matthes, CC BY 2.5)}};

\end{tikzpicture}


\end{document}